
%%%%%%%%%%%%%%%%%%%%%%% file typeinst.tex %%%%%%%%%%%%%%%%%%%%%%%%%
%
% This is the LaTeX source for the instructions to authors using
% the LaTeX document class 'llncs.cls' for contributions to
% the Lecture Notes in Computer Sciences series.
% http://www.springer.com/lncs       Springer Heidelberg 2006/05/04
%
% It may be used as a template for your own input - copy it
% to a new file with a new name and use it as the basis
% for your article.
%
% NB: the document class 'llncs' has its own and detailed documentation, see
% ftp://ftp.springer.de/data/pubftp/pub/tex/latex/llncs/latex2e/llncsdoc.pdf
%
%%%%%%%%%%%%%%%%%%%%%%%%%%%%%%%%%%%%%%%%%%%%%%%%%%%%%%%%%%%%%%%%%%%


\documentclass[runningheads,a4paper]{llncs}

\usepackage{amssymb}
\setcounter{tocdepth}{3}
\usepackage{graphicx}
%\usepackage{hyphenat} see
%http://www.ctan.org/tex-archive/macros/latex/contrib/hyphenat - would
%be good to enable this to prevent hyphenation of OWL MS statements.
%Or find alternative

\usepackage{url}
%\urldef{\mailsa}\path|davidos@ebi.ac.uk|

\newcommand{\keywords}[1]{\par\addvspace\baselineskip
\noindent\keywordname\enspace\ignorespaces#1}

\def\correspondingauthor{$^*$}
\def\@corresponding{\footnotesize\correspondingauthor Corresponding author} 


\begin{document}

\mainmatter  % start of an individual contribution

% first the title is needed
\title{Cell, chemical and anatomical views of the Gene Ontology.  Mapping to a Roche controlled vocabulary as a test case.}

% a short form should be given in case it is too long for the running head
\titlerunning{Cell, chemical and anatomical views of the Gene Ontology.}

% the name(s) of the author(s) follow(s) next
%
% NB: Chinese authors should write their first names(s) in front of
% their surnames. This ensures that the names appear correctly in
% the running heads and the author index.
%
\author{David Osumi-Sutherland$^1*$, ...}

%
\authorrunning{Cell, chemical and anatomical views of the Gene Ontology.}
% (feature abused for this document to repeat the title also on left hand pages)

% the affiliations are given next; don't give your e-mail address
% unless you accept that it will be published
\institute{$^1$ European Bioinformatics Institute (EMBL-EBI)
European Molecular Biology Laboratory
Hinxton, Cams, UK
%* corresponding author: davidos@ebi.ac.uk

% \mailsa\\  % This needs to be fixed!

%
% NB: a more complex sample for affiliations and the mapping to the
% corresponding authors can be found in the file "llncs.dem"
% (search for the string "\mainmatter" where a contribution starts).
% "llncs.dem" accompanies the document class "llncs.cls".
%

\toctitle{}
\tocauthor{}
\maketitle


\begin{abstract}

The Gene Ontology is part of a network of logically interconnected ontologies including ChEBI, the Cell Ontology and Uberon.  These logical interconnections make it possible to query the GO by cell type, chemical or anatomical structure, retrieving relevant GO terms and associated annotations.  In this paper we describe the use of such queries to automate mappings from a controlled vocabulary developed by Roche to lists of terms from the GO.

Using OWL-EL queries, we can fully automate mapping for about a third of terms in the Roche vocabulary, with another third having 5 or less GO terms requiring manual mapping.

The approach we describe here is not limited to mapping external vocabularies on to the GO. It could be used to provide chemical, cell or anatomically focussed ways of grouping GO annotations and of performing enrichment analyses. It could also be used for more sophisticated, combinatorial queries of the GO and its annotations.


\keywords{OWL, EL reasoning, ...}
\end{abstract}

\section{Introduction}


The Gene Ontology is widely used to annotate and group gene products according to their subcellular location (e.g., endoplasmic reticulum), molecular function (e.g., enzyme activity) and their wider role in cellular, developmental and physiological processes (e.g., signal transduction) (Gene Ontology Consortium, 2015). The logical structure of the ontology is used to group genes annotated with related terms and for term enrichment, a technique for determining the over- or under-representation of general classes of gene products in experimental datasets (Shah et al., 2012). Grouping and term enrichment typically only use logical relationships within each of the 3 sub-ontologies of the GO - cellular component, molecular function and biological process.

In recent years, GO has switched its underlying formalization to Web Ontology Language (OWL2) (http://www.w3.org/TR/owl2-primer/), and has dramatically increased the number of logical axioms (Mungall et al., 2014). This new axiomatisation includes many new relationship types, relationships between terms in different GO sub-ontologies and extensive logical links to terms from external ontologies including the cell ontology (Meehan et al., 2011), the chemical ontology ChEBI (Hastings et al., 2012) and the Uberon multi-species anatomy ontology (Mungall et al., 2012).  For example, the chemical participants in over 12,000 processes or functions are specified in the GO via axioms referencing chemical entities defined by ChEBI (Hill et al., 2013). Over 8000 GO classes have some direct or indirect logical link to a term from the Cell ontology or Uberon. These record, for example, the location of cellular components (e.g., the acrosome and its parts are present only in sperm), cell types that are the sole location of some process (e.g., 'natural killer cell degranulation' only occurs in natural killer cells ), and the products of developmental processes (e.g., bone is a product of 'bone morphogenesis').

Axiomatisation of the GO is limited to the EL profile of OWL (Mungal et al, 2014). This allows GO infrastructure to take advantage of fast, scalable OWL-EL reasoners such as ELK (Kazakov et al., 2012) to leverage the classifications in external ontologies to automate classification in GO, and to ensure that classification and querying of the GO will not become intractable as the ontology grows.  There is also great potential for using this axiomatisation to provide new, biologically meaningful systems for grouping annotations and term enrichment.  For example, we might want to group all annotations to genes involved in processes occurring in T-cells or in the pancreas, or to group annotations to genes involved in the processes involving nitric oxide.  In this paper we describe an implementation of this strategy in support of a use case from the pharmaceutical company Roche.

Roche uses a controlled vocabulary internally (from here referred to as RCV).  RCV consists of around 360 undefined terms, each of which is mapped to a set of terms from the GO. RCV includes terms named for biological processes and, more rarely, for molecular functions and cellular components.  It also includes many terms named for types of cell, chemical, anatomical structure and taxonomic group. Prior to this work, mappings from RCV to the GO were made manually, based on the lexical content of the names of GO terms and the biological knowledge of those doing the mapping.  As the GO evolved, it became increasingly impractical for Roche to keep this mapping complete and up-to-date via manual mapping.

Here we describe the development and testing of an automated mapping between GO and RCV, making use of OWL-EL reasoning and a standard system for specifying OWL design patterns.

\section{Results}

\subsection{Mapping strategy}

In the manual mapping between RCV and GO specified by Roche, multiple GO terms are mapped to each RCV term. For the purposes of automated mapping, we interpret the mapped GO terms as subclasses of the class referred to by the RCV term. For each RCV term, we attempted to find an equivalent class expression (a mapping query) that reflected the intended meaning of the RCV term, as judged by the RCV term name and manual mappings and based on discussion with Roche.

Mapping to fully expressive OWL-DL class expressions poses serious problems for scalability: querying and classification of the GO with OWL-DL reasoners is already slow, and may become intractable as the GO grows.  For this reason, we chose to restrict mappings to EL class expressions and use the fast, scaleable EL reasoner, ELK to run queries []. But this poses a problem for expressiveness: EL lacks various elements of OWL that are potentially useful for mapping queries - most notably disjunction (OR) and negation (NOT).  

To compensate partially for the lack of disjunction in OWL-EL, we developed a set of high level object properties for use in queries. For example, we define occurs_in_OR_has_participant as a grouping relation, allowing queries for processes that occur in a specified cell, or have that cell as a participant. Similarly, GO does not use a reflexive relation for 'part of', but using one for query purposes means a query for subclasses of  "'part of' some X" returns both subclasses and proper parts of class X.

Many RCV terms group processes in which a specified chemical or cell participates, with processes regulating those in which it participates (see Table 1 for example). To support such groupings, we used an OWL property chain axiom (http://www.w3.org/TR/owl2-primer/#Property_Chains) to define a relation, regulates_o_has_participant, to query for processes that regulate a process in which some specified entity is a participant. We then define a super-property, participant_OR_reg_participant, for this new relation and has_participant



\textbf{regulates} \textit{o} \textbf{has_participant} \textit{subPropertyOf}
\textbf{participant_OR_reg_participant}
\textbf{regulates_o_has_participant} \textit{subPropertyOf} \textbf{participant_OR_reg_participant}
\textbf{has_participant} \textit{subPropertyOf} \textbf{participant_OR_reg_participant}


In order to keep the mapping process simple, we added a further restriction: only a single mapping class was specified for each mapping.  

The heavy use of OWL Object Property axioms to compensate for loss of expressivity tends to obscure the semantics of mappings. In order to communicate the meanings of mappings clearly, we used a script to generate human readable descriptions for each mapping query.  For example, we mapped the RCV term cannabinoid to the ChEBI term cannabinoid (CHEBI:67194) plus pattern participant_OR_reg_participant.  The automated description of the mapping reads:  "A process in which a cannabinoid participates, or that regulates a process in which a cannabinoid participates."

Each mapping was used to generate a mapping table for manual review (see Table 1 for example), allowing the possibility of blacklisting either automated mappings or manual mappings (as a way of specifying corrections to the original manual mapping).


\section{Discussion and future directions}

This work demonstrates how the logical structure of the GO can be used to achieve biologically meaningful mappings between GO and terms from external controlled vocabularies or ontologies for which there is no corresponding GO term.  For example, where the external vocabulary refers to a cell-type, a chemical or an anatomical structure.  The mapping system used is fast and scalable,  

Improving the RCV mapping pathway

There is good scope for improving the mapping between RCV and GO so that it is more thoroughly automated. As shown in table 2, the mapping between RCV term names and mapping patterns is reasonably consistent, but, there are exceptions to these mappings.  More consistency in naming will make the meaning of terms more predictable and make the mapping of new terms more straightforward. 

For RCV terms with only a small number of additional manual mappings, it may be worth considering whether the overhead of manual maintenance is worth the effort, especially where these additional mappings could not be achieved by further axiomatisation of the GO.  For example, mappings to cell types often include mappings to growth factors acting on those cell types.  As these growth factors have much broader functions than action on the cell types for which they are named, GO is unable to add any formal link between factors and cell types.

In other cases a mapping pattern involving two or more specified classes and a more sophisticated logic would be necessary to obtain a complete mapping.  For example, the manual mappings for X metabolism terms are consistently mapping to bot X metabolism and X transport terms in the GO. A more complete mapping to RCV metabolism terms could be achieved using a pattern that named both GO transport and GO metabolic process terms.  This could be made scalable with a pathway that combines the results of multiple mapping patterns outside of OWL.

56 terms were not mapped.  Some were rejected from the pipeline as they were judged to be duplicates with other RCV terms.  The rest were rejected as currently unmappable due to the lack of suitable terms or axiomatisation within the GO at this time.  For example, GO currently has no way to group aerobic or anaerobic metabolic processes, although it does reflect the aerobic or anaerobic nature of many metabolic processes in their names and textual definitions. Further formalisation of the GO is likely to improve the number of concepts that can be mapped.

\subsection{Alternative views of the GO and its annotations}

The mechanisms described here for mapping to external ontologies could also be used for providing alternative views of the GO and its annotations.  This is already reflected in some of the newer functionalities of the GO browsing tool AMIGO, which now displayed inferred annotations to cell-types based on axioms in GO recording where processes occur.  Figure 3 shows an AMIGO display of annotations to gene involved in processes occurring in T-Cells

\subsection{Future work}

The system described here was designed to be lightweight and flexible, allowing maximum interaction between the designers of RCV at Roche and GO editors with minimal development overhead.

The pattern-based system used here bears some relationship to the TermGenie system (Dietze et al., 2015) which is already used to generate 80\% of new GO terms.  One possible approach to fulfilling the needs of external groups for types of classification not included in the GO would be to offer a TermGenie-like system for generating terms that group GO terms in ways that are not currently supported internally by the GO.


\section{Methods}

All code, mapping tables and results for the pipeline were maintained in a GitHub repository (https://github.com/GO-ROCHE-COLLAB/Roche_CV_mapping). As well as providing version control, Github allows nicely formatted display of mapping and results files in an easily editable form (tab separated value (TSV), which can be easily edited via copying and pasting from excel spreadsheets). It also has an integrated ticket system, with an open API.  Standard mapping tickets were generated by script for all RCV terms mapped. A standard system of ticket labels allowed tracking of the approval status of all mappings.  The archived tickets constitute an audit trail for approval of mappings.

The mapping was specified using a single TSV file in which each line maps an RCV term to a mapping query and a term from GO, ChEBI, CL, Uberon or NCBI taxonomy.

OWL reasoning was carried out via calls to a standard Java API for OWL using the ELK OWL reasoner (Kazakov et al., 2012).  The query and processing pipeline was written in Jython, a Python implementation over Java (http://www.jython.org/).  The pipeline produced a set of results tables, one for each RCV term, in TSV format.  These were used for review of mappings by Roche. 


\begin{figure}
\centering
\includegraphics[width=120mm]{images/Query_menus_DL_images.png}
\caption{VFB query menus (left) with the DL queries they run (right).  The top panel
  shows nested queries for classes of neurons based on \textbf{overlaps} and its
  subproperties.  Each of these queries returns classes based on
  assertions further down the partonomy than the query term.  The
  bottom panel show queries for individual neurons with and without
  clustering.  A 3D rendering of a cluster is shown on the bottom right. }
\label{fig:Query_menus_DL_images}
\end{figure}



\begin{figure}
\centering
\includegraphics[width=120mm]{images/combined_explanation.png}
\caption{\textbf{A.} Explanation for why the query ``\textit{not}
   \textbf{has\_part} \textit{some} `fan-shaped body' '' returns
   `antennal lobe'.  Note that direct assertion of \textbf{has\_part}
   restriction axioms is not necessary. \textbf{B.} Explanation for
   why the query ``neuron that (\textbf{has\_synaptic\_terminal\_in}
   \textit{some} `antennal lobe') and not
   (\textbf{has\_synaptic\_terminal\_in} \textit{some} `fan-shaped
   body')'' returns the neuron `DL1 adPN'. }
\label{fig:combined_explanation}
\end{figure}

\begin{figure}
\centering
\includegraphics[width=120mm]{images/expression_pattern_with_neg.png}
\caption{The left panel shows the expression pattern of the
  P{GMR11H01-GAL4} transgene in the adult brain.  The right panel
  shows its representation in OWL. Only one explicit negation is shown,
  but the full OWL representation includes negative expression
  assertions for 30 brain regions.  These explicit negations are
  necessary in the absence of sufficient information to add closure axioms.}
\label{fig:exp_pat_neg}
\end{figure}








\subsection*{Author's contributions}


\subsection*{Acknowledgments.}

\subsubsection*{Funding}

This work was supported by direct funding from Roche.  The Gene Ontology is supported by ...

%\begin{thebibliography}{}

\bibliography{SWAT4LS.bib} % Bibliography file (usually '*.bib' )

\bibliographystyle{plain}

%\end{thebibliography}

\end{document}
